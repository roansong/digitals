\section{Conclusions}
OpenCL is evidently a powerful tool for developing unified multi-platform computing solutions. Given different platforms with which to work, it is important to understand the strengths and capabilities of each, in order to allow them to be applied to tasks suited to them. This report has found that the NVIDIA GPU is suited to coarse parallel tasks and offers significant speed-up over the Intel CPU with which it is paired. This speed-up is proportional to the size of the data at hand. If working on datasets of less than 10,000 items, the average speed-up is 1.5, which may not justify an investment into the more expensive high-end GPU.  
\newline Consideration needs to be taken towards the predominant type of computations run on the system, as this will change the point at which the speed-up is effected by larger data sets.  These trade-offs can be estimated when designing a HPEC system, but it is recommended to sample a greater number of datasets as well as a more diverse range of kernels; accurate depiction of the trade-off points for various hardware types will help to optimize the systems' performance at minimal cost.